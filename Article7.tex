\chapter{Housing Policies}

\section{Alcohol, Tobacco, and Illegal Substances}

	\begin{enumerate}
		\item No alcohol is to be consumed in the house or on the chapter property. \label{alcohol}

		\item No alcohol is to be stored in the house or on the chapter property except for the purposes of cooking. \label{alcohol-storage}
			\begin{enumerate}
				\item Alcohol stored for cooking may only be stored for a period of twenty-four hours.
			\end{enumerate}

		\item Tobacco may be possessed and chewed within the house or on the chapter property. \label{tobacco} 

		\item Illegal substances may not be possessed, stored, or used in the house or on the chapter property. \label{substances}

		\item If a brother or his guest violates \hyperref[alcohol]{Article VII, Section A, \autoref*{alcohol}}, \hyperref[alcohol-storage]{Article VII, Section A, \autoref*{alcohol-storage}}, \hyperref[tobacco]{Article VII, Section A, \autoref*{tobacco}}, or \hyperref[substances]{Article VII, Section A, \autoref*{substances}}, he will be brought up for standards by the President, with sanctions not less than: 
			\begin{enumerate}
				\item First Offense --- \$50 fine
				\item Second Offense --- \$75 fine, placed on \gls{prob}, and becomes \gls{staggard}
				\item Third Offense --- \$100 fine and is recommended for suspension to the chapter
			\end{enumerate} 
		
		\item No candles will be burned inside the house or on the chapter property. 

		%Should the downfall of Western Civilization occur, you should immediately acquire Rummy Bears and then acquire drunk bitches.
        %The former can be created by placing a large amount of rum in a bowl, then placing Gummy Bears in that bowl and leaving them to soak overnight.
        %This author has no idea where to find the latter.
		\item \hyperref[alcohol]{Article VII, Section A, \autoref*{alcohol}}, \hyperref[alcohol-storage]{Article VII, Section A, \autoref*{alcohol-storage}}, and \hyperref[tobacco]{Article VII, Section A, \autoref*{tobacco}} will become null and void upon the downfall of Western Civilization.
	\end{enumerate}

%Housing will work as prescribed unless someone is not screwed over. If nobody is screwed over, the Vice-President will construct housing to screw over the maximum number of people.
\section{Housing Eligibility}

	\begin{enumerate}
		\item All active brothers are to live in the chapter house pursuant to Article V, Section 10 of the \gls{ibl}.\label{in-house}
			\begin{enumerate}
				\item The following brothers are exempt from this policy:
					
					\begin{enumerate}
						\item 1st year students
						\item Commuter Students
						\item Those who are married
						\item Those who receive housing as compensation for employment (e.g. Resident Assistants)
						\item Those who are on Co-op, Study Abroad, or a similar program
						\item Those who were ineligible or exempt at the beginning of the school year
                        \item Those with a prohibitive medical condition as approved by Case Western Reserve University
					\end{enumerate}
			\end{enumerate}

		\item The following brothers must live in the chapter house: \label{ec-in-house}
			\begin{enumerate}
				\item President
				\item Vice President
				\item Vice President of Health and Safety
				\item Treasurer
				\item Marshal
				\item Recruitment Chairman
				\item House Manager
				\item Food Stewart
			\end{enumerate}

		\item Brothers in violation of \hyperref[in-house]{Article VII, Section B, \autoref*{in-house}} or \hyperref[ec-in-house]{Article VII, Section B, \autoref*{ec-in-house}} are to be brought up for standards by the Vice President with the possibility of suspension.

		\item In order for a brother to move in to the house, he must not be on \gls{fin-probation} and must not have a debt to the chapter exceeding \$3000, including \glspl{defer}.
        \begin{enumerate}
            \item Brothers living in the house during the fall of any year are exempt from this stipulation for the spring semester of that year only.
        \end{enumerate}

		\item Brothers living in the house must have a signed housing contract by the due date (\hyperref[housing-contract]{Article V, Section D, \autoref*{housing-contract}}). 

		\item Alumni will be permitted one year after graduation or status change to live in the house without chapter approval. After that semester, the brother in question must receive a \gls{super-maj} vote in order to remain in the house. %Case policy says that a person must be taking classes to live in the house. Alums living in must be grad students. Students who have dropped out are in a kind of grey area.
			\begin{enumerate}
				\item Alumni must be taking classes at \gls{cwru} to live in the house (i.e. as a graduate student).
			\end{enumerate}

		\item Only a spouse to a brother living in the house and taking classes at \gls{cwru} may live in the house during the school year. A spouse meeting these qualifications must receive a \gls{super-maj} vote of the chapter to live in the house.

		\item Active brothers will have priority over all other possible residents for the purposes of living in the house and room assignments, at the discretion of the Vice President.
	\end{enumerate}

\section{Room Assignments}
\label{housing}

	\begin{enumerate}
		\item If the number of brothers eligible to live in the house exceeds the capacity of the house:
			
			\begin{enumerate}
				\item Brothers who have lived in the house during the school year for at least one semester will be allowed the option of moving out in order of pledge class from oldest to most recent, at the discretion of the Vice President. \label{move-out}

					\begin{enumerate}
						\item The house may not be dropped below capacity.

						\item If this method would bring the house below capacity: 
							\begin{enumerate}
								\item Of the brothers with the highest \gls{roster} among those who wish to move out, select the number of brothers that would bring the house to capacity by a mutually acceptable method.
							\end{enumerate}
					\end{enumerate}

				\item If, after application of the method described in \hyperref[move-out]{Article VII, Section C, \autoref*{move-out}}, the house is still above capacity, brothers who have lived the most semesters in the house (excluding summer) will be forced to move out.
					
					\begin{enumerate}
						\item If two or more brothers have lived in the house the same amount of time, the one with the lowest unmodified \gls{roster} will be forced to move out first.

						\item Brothers listed in \hyperref[ec-in-house]{Article VII, Section B, \autoref*{ec-in-house}} cannot be forced to live out of the house.
					\end{enumerate}
			\end{enumerate}

		\item Any brother applying for a room must have less than \$50 debt, including \glspl{defer}, at the time room assignments are made, lest he become \gls{staggard}.

		\item Room 307 is reserved for the President unless he waives this right.

		\item Room assignments will be made by the Vice President following these guidelines: \label{housing-guidelines}
			\begin{enumerate}
				\item If two or more groups of brothers wish to occupy the same room, the room will be assigned to:

					\begin{enumerate}
						\item The group with the most brothers, not to exceed the capacity of the room; else
						\item The group of brothers with the greatest number of brothers with \gls{squatters} for that particular room; else
						\item The group of brothers with the lowest aggregate \gls{roster}; else
						\item The group of brothers decided by a mutually acceptable method.
					\end{enumerate}
			\end{enumerate}

		\item The Vice President will have penultimate authority, beneath the chapter, to assign rooms to the best of his ability in keeping with the guidelines established in \hyperref[housing-guidelines]{Article VII, Section C, \autoref*{housing-guidelines}}.

	\end{enumerate}
