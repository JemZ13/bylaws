
\chapter{Finances}

\section{Duties of the Treasurer}
	\begin{enumerate}
		\item The Treasurer will perform all duties outlined in \hyperref[treasurer-duties]{Article V, Section D, \autoref*{treasurer-duties}}. 
	\end{enumerate}

\section{General Rules}
	\label{fin-rules}
	\begin{enumerate}
		\item Two signatures are required on all accounts of the chapter, pursuant to the \gls{ibl}, Article V, Section 9, paragraph b.
		\item Neither the chapter nor alumni corporation can have or use any form of ATM, debit, credit or money card that is attached to any account of the chapter or alumni corporation.
		\item An officer cannot issue a check to himself.
		\item The chapter will not utilize a petty-cash system. All transactions must be made by
		check.
		\item All reimbursements require a receipt.
		\item All statements and record books are reconciled monthly.
        \item Only the President or Treasurer may sign contracts on behalf of the chapter.
		%These rules come from the rules to get embezzlement insurance from IHQ.
	\end{enumerate}

\section{Budget}
	\label{budget}
	\begin{enumerate}
		\item The Treasurer will present a preliminary budget to the Budget Committee (\hyperref[budget-committee]{Article V, Section G, \autoref*{budget-committee}}) no later than six weeks before the end of the Fall Semester.
		\item The Treasurer will submit the final budget proposal, as approved by the Budget Committee, for the following academic year to the chapter no later than three weeks before the end of of the Fall Semester. The proposal is approved by \gls{super-maj} vote of the chapter.
			\begin{enumerate}
				\item Following approval by the chapter, the budget proposal will be presented to the Alumni Corporation and the Alumni Financial Advisor.
			\end{enumerate}
            

		\item Budget Committee
			\begin{enumerate}
				\item The Budget Committee (\hyperref[budget-committee]{Article V, Section G, \autoref*{budget-committee}}) will be created and convened no later than six weeks before the end of the Fall semester.
				\item The committee will create a budget to be submitted to the chapter based on the preliminary budget created by the Treasurer.
			\end{enumerate}
          
        \item Any officer with a budget may distribute portions of his budget as he sees fit.
	
    
    	\item Safety Marigin and Emergency Fund
        	\begin{enumerate}
            	\item Any money leftover for the year from the Safety Margin must be moved to an emergency fund account until that account has 50\% of the next year’s projected expenses saved. This account is only to be accessed with the vote of EC and the CAB.
             \end{enumerate}
     \end{enumerate}

\section{Financial Obligations of Members}
\label{fin-obligations}

	\begin{enumerate}
		\item Each brother, New Member, and Alumnus is responsible for any fraternity property damaged, broken, or lost due to any negligence or carelessness on the part of that person or his guest. 
		\begin{enumerate}
			\item Any decision regarding this rule will be decided by \gls{ec}.
		\end{enumerate}

		\item Brothers will pay all dues and other financial obligations to the chapter as defined in the budget.

	\end{enumerate}

\section{Deferments}
\label{def}
	\begin{enumerate}
		\item If a brother deems that he will be unable to pay his bill by the due date, he may submit a \gls{def}
		\item A \gls{def} must be submitted to \gls{ec} by the due date of the bill that is being deferred. 
		\item A \gls{defer} must be approved by a vote of \gls{ec} or a \gls{super-maj} vote of the chapter.
		\item A brother that is not following the payment plan in his \gls{defer} will have his account balance considered past due.
		\item Interest on late payments may be charged at the discretion of \gls{ec}.
	\end{enumerate}

\section{Payment Schedule and Penalties}

	\begin{enumerate}
		\item The due date for any bill will be at least two weeks after the date on which the bill was issued and will be before the next bill is to be issued.

		\item The chapter may, by unanimous ballot, exempt a brother from payment of any debts owed to the chapter.
			\begin{enumerate}
				\item The unpaid debt will be divided equally among the other members of the chapter.
			\end{enumerate}

		\item If the account balance of any brother, New Member, or Alumnus is negative and past due, he will be placed on \gls{fin-probation} unless a \gls{defer} has been approved.
		\label{fin-probation}
			\begin{enumerate}
				\item A brother placed on \gls{fin-probation} will be fined no more than \$20.
				\item A brother on \gls{fin-probation} loses voting rights.
				\item If a brother is on \gls{fin-probation} for more than one week, a suspension trial will be scheduled for him.
				\item The Alumni Corporation and the chapter financial advisor will be notified of any brother placed on \gls{fin-probation}.
			\end{enumerate}

		\item If a brother has a debt, including \glspl{defer}, of over \$3,000:
			\begin{enumerate}
				\item \Gls{ec} will vote on whether or not to hold a suspension trial for him after every bill.
				\item The Treasurer will announce the brother's status to the chapter after every bill.
				\item Any receipt forms that the brother submits must be credited against that debt unless \gls{ec} approves an exception.
			\end{enumerate}
            \end{enumerate}
\section{Finance Privacy Section}
	
    \begin{enumerate}
    	\item Executive Council and CAB will have access to all accounts, including bank accounts, budgeting and financial software, the current ledger, and Omega Fi(or whatever online service is currently being used).
    \end{enumerate}
\section{Reimbursements}
	\begin{enumerate}
    	\item To buy something(s) for an event that total more than \$250 and obtain a reimbursement, you need the treasurer’s approval and the respective chair’s approval. If you are one of those people, you must get the approval from the president instead of yourself. .
    \end{enumerate}
