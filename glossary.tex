%This file contains glossary entries.
%Define terms using \newglossaryentry{key}{name=Name,description={Definition}}
%Key is a shortname that is for internal use.
%Reference them with \gls{key}, \Gls{key}, \glspl{key}, or \Glspl{key} for normal, first letter capitalized of each word, plural and plural first letter capitalized.

%NOTE: Entries have a period entered at the end automatically.


\newglossaryentry{staggard}{name=staggard,description={A brother who is staggard loses \gls{squatters} and has 100 added to his \gls{roster}. This effect stacks with itself}}
%Explanation: While revising the bylaws in 2010, Matthew Richter had been writing some things on a white board referring to officer terms, if I recall correctly, and meant to write that the terms should be ``staggered". Instead, he wrote ``staggard". Since the condition now known as staggard was peppered throughout the bylaws, we wanted to give it a term so that the bylaws could be shortened. Someone suggested staggard since we had been making fun of Richter for it and we decided to keep it. Later, we found out that it is a real word. From the American Heritage dictionary, "A male red deer in its fourth year." It is middle English from "Stagge" which is related to "Stag".

\newglossaryentry{squatters}{name=squatter's rights,description={The second highest proiority given to a brother when deciding housing assignments. A brother has squatter's rights for a room if he is living in that room during the semester in which room assignments are made}}

\newglossaryentry{roster}{name=roster number,description={For the purposes of these bylaws, a brother's roster number starts as the median of the actual roster numbers of his New Member Class}}

\newglossaryentry{social}{name=social event,description={Open and closed parties sponsored by the fraternity and any other event given this designation by \gls{ec} excluding house meals and Rush events. Examples: Mixers, some Greek Week Events, and Formal}}

\newglossaryentry{schol-program}{name=scholarship program,description={The semesterly legislation created by the Scholarship Chairman which is to include scholarship goals for the chapter and means or reaching those goals}}

\newglossaryentry{abs-maj}{name=absolute majority,description={A vote of over 50\% of all the voting body, counting all brothers, including those not present}}

\newglossaryentry{simp-maj}{name=simple majority,description={A vote that must be over 50\% of those brothers present and eligible to vote to succeed providing \gls{qorm} is present}}

\newglossaryentry{super-maj}{name=supermajority,description={A vote that must be over $\frac{2}{3}$ of the voting body to succeed}}

\newglossaryentry{grieve}{name=grievance,description={A request submitted to \gls{ec} for clarification of the intention of the bylaws}}

\newglossaryentry{qorm}{name=quorum,description={One more than half of the active chapter or the body that it describes}}

\newglossaryentry{balloting}{name=ballot,description={A secret vote as defined in \hyperref[balloting]{Article IV, \autoref*{balloting}}}}

\newglossaryentry{attend}{name=attend,description={Attendance is at the discretion of the organizer of the event, with a suggested attended portion being 3/4 of the event's total run time}}

\newglossaryentry{special-detail}{name=special event detail,description={An \gls{official-chapter-event} declared by the Detail Manager with a stated objective such as cleaning the house before inspections or a Rush event. The chapter must be given at least twenty-four hours notice of such a session and excuses for the session will be approved by EC.}}

\newglossaryentry{awesome}{name=Awesome,description={A group of four or more brothers of Theta Chi}}

\newglossaryentry{res-plea}{name=responsible plea,description={The Accused admits that he is rightfully accused and agrees to all sanctions given}} %This is a thing that I did. 

\newglossaryentry{not-res-plea}{name=not responsible plea,description={The Accused denies culpability for the accusation}} %I didn't do it!

\newglossaryentry{no-cont-plea}{name=no contest plea,description={The Accused attests to the truth of the accusation, but believes there are mitigating circumstances surrounding the accusation which exculpate him}} %"Hey Man! It's not my fault!"

\newglossaryentry{not-resp}{name=not responsible,description={A ruling that states the Accused is not culpable for the accusation against him}}

\newglossaryentry{resp}{name=responsible,description={A ruling that states the Accused is culpable for the accusation against him}}

\newglossaryentry{fin-probation}{name=financial probation,description={See \hyperref[fin-probation]{Article VI, Section F, \autoref*{fin-probation}}}}

\newglossaryentry{grade-imp-plan}{name=grade improvement plan,description={A plan made by a brother on \gls{academic-watch} with help from the Scholarship Chairman to improve his grades}}

\newglossaryentry{prob}{name=probation,description={A brother on probation cannot attend social events and loses his voting rights}}

\newglossaryentry{official-chapter-event}{name=official chapter event,description={An event that all brothers must attend}}

\newglossaryentry{req-def}{name=request for deferment,description={A form that includes how much money a brother wishes to defer, the reason a \gls{defer} is needed, and a payment plan for the amount deferred},plural={Requests for Deferment}}

\newglossaryentry{defer}{name=deferment,description={A payment plan approved by \gls{ec} based on a \gls{def} and the recommendations of the Treasurer for a brother's debt}}

\newglossaryentry{academic-probation}{name=Academic Probation,description={See \hyperref[academic-probation]{Article X, Section B, \autoref*{academic-probation}}}}

\newglossaryentry{academic-watch}{name=Academic Watch,description={See \hyperref[academic-watch]{Article X, Section B, \autoref*{academic-watch}}}}

\newglossaryentry{pytte}{name=Community Standards,description={The list of requirements set forth by the Greek Life Office at \gls{cwru} which is to be completed yearly in the fall semester}}
% I did some jank shit here by renaming the word used to refer to pyttecup/community standards when this glossary entry is referenced, but not changing the handle in the glossary to save a huge amount of time looking for every time \gls was used to refer to {pytte}.

\newglossaryentry{safety}{name=Safety Margin,description={A sum of money built into the budget for the chapter which is set aside in case of unexpected expenses.}}

\newglossaryentry{ex-ed}{name=Executive Edict,description={Legislation implemented in a time of emergency by a supermajority vote of EC. See \hyperref[executive-edict]{Article X, Section B, \autoref*{executive-edict}}}}

\newglossaryentry{plur}{name=plurality,description={The number of votes cast for a candidate who receives more than any other but does not receive an absolute majority.}}


%Here is where acronyms start.
%\newacronym{label}{short form}{expanded form}
\newacronym{cab}{CAB}{Chapter Advisory Board}

\newacronym{ccc}{CCC}{Committee Chair Committee}

\newacronym{ibl}{IBL}{International Bylaws}

\newacronym{ec}{EC}{Executive Council}

\newacronym{ifc}{IFC}{Interfraternity Congress}

\newacronym{cwru}{CWRU}{Case Western Reserve University}

\newacronym{smarrt}{SMARRT}{Students Meeting about Risk and Responsibility Training}

\newacronym{gpa}{GPA}{Grade Point Average}

\newacronym{ihq}{IHQ}{International Headquarters}

\newacronym{cab}{CAB}{Chapter Alumni Board}
