\chapter{Standards Board}
\label{stds-board}

\section{Purpose}
	\begin{enumerate}
		\item To judge infractions of the bylaws and standards of the chapter.
        \item To recognize the achievements of chapter members and offices.
        %\item To protect the world from devastation
        %\item To unite all peoples within our nation
        %\item To denounce the evils of truth and love
        %\item To extend our reach to the stars above
	\end{enumerate}

\section{Membership}

	\begin{enumerate}
		\item The membership of the Standards Board is seven Justices; the First Guard, who presides as Arbiter; the Second Guard, who serves as Scribe; and the Vice President, who serves as Parliamentarian.
        \item Justices are elected according to the procedure in Article IV, Section B, Item 4.
        \item The Arbiter and Scribe are ineligible to serve as Justices on the board.
		\item Members of \gls{ec} are ineligible to serve as Justices on the board.
		\item Justices must at all times be eligible to hold office (\hyperref[officer-eligibility]{Article X, Section B, \autoref*{officer-eligibility}}). 
		\item Justices serve on the board until resignation, ineligibility, removal from office, or graduation. \label{justice-term}
	\end{enumerate}

\section{Voting}

	\begin{enumerate}
		\item Unless otherwise specified, votes of the board are by \gls{simp-maj}.
		\item All Justices are voting members of the board.
		\item The Scribe and Parliamentarian are non-voting members of the board and the Arbiter will only vote in the case of a tie.
		\item Quorum of the board consists of $\frac{2}{3}$ of its members, including at least $\frac{2}{3}$ of the active Justices.
	\end{enumerate}

\section{Duties of Membership}

	\begin{enumerate}

		\item Arbiter
			\begin{enumerate}
				\item Schedules and presides at all meetings, hearings, and academic reviews.
				\item Reviews all claims with the Scribe to ascertain their validity.
				\item Announces final rulings to the chapter. \label{final-rulings}
				\item Keeps confidential all matters relating to judicial proceedings with the exception of announcing rulings as in \hyperref[final-rulings]{Article XII, Section D, \autoref*{final-rulings}}. 
				\item Treats all brothers fairly, impartially, and consistently.
				\item Will ensure that all Justices receive training on judicial procedure and relevant rules and policies each semester.
				\item Advises the board on matters of precedent.
				\item Ensures the completion of all sanctions.
			\end{enumerate}

		\item Scribe
			\begin{enumerate}
				\item Keeps records of all meetings and hearings of the board.
				\item Reviews all claims with the Arbiter to ascertain their validity.
				\item Keeps confidential all matters relating to judicial proceedings.
				\item Treats all brothers fairly, impartially, and consistently.
				\item In the absence of the Arbiter, arbitrates.
			\end{enumerate}

		\item Justice
			\begin{enumerate}
				\item Hears and votes on all matters brought before the board.
				\item Attends all meetings of the board.
				\item Keeps confidential all matters relating to judicial proceedings.
				\item Treats all brothers fairly, impartially, and consistently.
				\item Attend judicial training once per school year 
			\end{enumerate}

		\item Parliamentarian
			\begin{enumerate}
				\item Advises the board on matters of procedure.
				\item Answers any questions concerning the \gls{ibl}, \gls{fipg} guidelines, \gls{ifc} bylaws, Greek Life policies, university regulations, local bylaws, or legislation.
				\item Serves as liason between \gls{ec} and the board.
			\end{enumerate}	

		\item If the Arbiter, Scribe, or Parliamentarian is absent, has been recused, or is the accused, the order of succession will be:
			\begin{enumerate}
				\item Arbiter
				\item Scribe
				\item Parliamentarian
				\item Chapter President
				%If it goes beyond this, the chapter is in fucking problems.
			\end{enumerate}

	\end{enumerate}

\section{Justice Removal}
	\begin{enumerate}
		\item If, at any time, a Justice becomes ineligible to hold office or is elected to an \gls{ec} position, that Justice is removed from office.
		\item The removal of a Justice for reasons other than ineligibility to hold office will follow the procedure for a hearing with the following exceptions:
			\begin{enumerate}
				\item The accused Justice will not be counted as a member of the board.
				\item The only possible sanction is removal from office.
				\item The decision will require a \gls{super-maj} vote of the board.
			\end{enumerate}

		\item Any Justice may be removed for the following reasons:
			\begin{enumerate}
				\item Failure to perform his duties, including unexcused abscence from more than one board meeting per semester.
				\item Failure to maintain confidentiality of all proceedings of the board.
				\item Conduct unbecoming of a Justice or which negatively affects the credibility of the board.
			\end{enumerate}
	\end{enumerate}

\section{Charges}

	\begin{enumerate}
		\item Any brother may be brought up for standards for any of the following:
		
		\begin{enumerate}
			\item Violation of the \gls{ibl}.
			\item Violation of the local bylaws.
			\item Violation of the Expectations of Brotherhood (\autoref{expectations-bhood}).
			\item Failure to comply with a previous sanction.
			\item Conduct unbecoming of a brother.
		\end{enumerate}

	\end{enumerate}

\section{Accusation Procedure}

	\begin{enumerate}
		\item All accusations must be submitted in writing to the Arbiter. 
			\begin{enumerate}
				\item Each accusation must be as detailed as possible including at least the clause of the local bylaws that was violated.
				\item Each accusation must be submitted within one month of the violation.
			\end{enumerate}

		\item Upon receiving the Accusation, the Arbiter and Scribe, with the assistance of the Parliamentarian, will determine the validity of the claim.
		\item The Arbiter will inform the accuser of the validity of the accusation.
		\item If the accusation is found to be without merit, proceedings will end and no hearing will be held.
		\item If the accusation is found to have merit, the Arbiter will provide written notification to the accused.
			\begin{enumerate}
				\item The notification must be given within forty-eight hours of receipt of the accusation.
				\item The notification must include the charges with enough detail to allow the accused to prepare a defense.
			\end{enumerate}
		\item Upon receiving the notification, the accused has twenty-four hours to submit a plea in writing to the Arbiter.
			\begin{enumerate}
				\item The possible pleas are \gls{res-plea}, \gls{not-res-plea}, and \gls{no-cont-plea}.
				\item If a plea is not received, a \gls{no-cont-plea} will be entered.
			\end{enumerate}

		\item After a plea is entered, the Arbiter will schedule a time and place for the hearing.
			\begin{enumerate}
				\item The hearing will take place between seven and twenty-two days after a plea is entered.
				\item The Arbiter will notify the accused, accuser, and the board of the time, place, and nature of the hearing.
			\end{enumerate}

		\item The Parliamentarian will inform \gls{ec} of the accusation, the accused, and the nature of the complaint at the next \gls{ec} meeting.

	\end{enumerate}

\section{Hearing Procedure}

	\begin{enumerate}
		\item Recusal
			\begin{enumerate}
				\item Members of the board will recuse themselves from the hearing if they feel a conflict of interest exists or they are the accused.
				\item Members of the board may be recused by \gls{simp-maj} vote of the remaining board. The accused, accuser, or a member of the board may call for such a vote.
			\end{enumerate}

		\item Procedure for a \gls{not-res-plea}:
			\begin{enumerate}
				\item Preparation
					\begin{enumerate}
						\item Board members, the accused, and the accuser will be in badge attire.
						\item Both the accuser and the accused may provide witnesses.
						\item Witnesses will be advised by the Arbiter that they are not to discuss the hearing or the accusation outside of the hearing.
						\item The Arbiter will call the hearing to order and ask all but the directly related participants to leave the room. The directly related participants are the Arbiter, Scribe, Parliamentarian, Justices, accuser, and accused. 
						\item Witnesses must remain outside the room until called by the Arbiter and leave when dismissed.
					\end{enumerate}
				\item Method
					\begin{enumerate}
						\item The Arbiter then reads aloud the charges and ensures that the accused understands the charges.
						\item The accuser and accused may then give an opening statement, in that order.
						\item The accuser may then be asked questions by the accused and the board.
						\item The accused may then be asked questions by the accuser and the board.
						\item The accuser may then present witnesses. If he is not present, the Arbiter may call witnesses. After the witness gives his statement the accuser, accused, and board may ask questions of the witness. The accuser may call himself as a witness.
						\item The accused may then present witnesses. After the witness gives his statement, the accuser, accused, and board may ask questions of the witness.
						\item The accuser and accused may give closing statements, in that order.
						\item The Arbiter will excuse the accuser and the accused from the hearing.
						\item The board will then deliberate.
							\begin{enumerate}
								\item The board will consider the facts, evidence, and testimony presented in the case. 
								\item The board will not consider past judicial history of the accused in determining if the accused is \gls{resp}.
							\end{enumerate}
						\item The board will then vote on whether the accused is \gls{resp}.
						\item If the accused party is found \gls{resp}, the board must then determine an appropriate sanction.
							\begin{enumerate}
								\item Past judicial history of the accused and sanctions given for similar accusations in the past may be considered.
								\item The board will vote on the sanction.
							\end{enumerate}

						\item If the accused is found \gls{not-resp} or after the sanctions have been determined the Arbiter will readmit the accuser and accused back to the hearing and will announce the results of the hearing as well as any sanctions given.
						\item The Arbiter will then inform the accused and accuser of the appeal process.
						\item The Arbiter will then adjourn the hearing.
					\end{enumerate}

			\end{enumerate}

		\item Procedure for a \gls{res-plea} or a \gls{no-cont-plea}:
			\begin{enumerate}
				\item The dress at the hearing will be informal.
				\item The Arbiter will call the hearing to order and ask all but the board and the accused to leave the room.
				\item The Arbiter will read aloud the charges and ensures that the accused understands them.
				\item The accused may give a statement.
				\item The board may question the accused.
				\item The Arbiter will excuse the accused.
				\item The board must determine an appropriate sanction.
					\begin{enumerate}
						\item Past judicial history of the accused and sanctions given for similar accusations in the past may be considered in determining a sanction.
						\item The board will vote on the sanction.
					\end{enumerate}

				\item The Arbiter will readmit the accused and announce the sanction.
				\item The Arbiter will inform the accused of the appeal process.
				\item The Arbiter will adjourn the hearing.
			\end{enumerate}

		\item Following the hearing, the Arbiter will provide written notification of the results and sanctions, if any, to the accused within twenty-four hours of the hearing.
		\item If the accused is found \gls{resp} the Arbiter will inform the chapter of the name of the accused, the accusation, and the sanctions at the next regularly scheduled chapter meeting.
		\item If the accused is found \gls{not-resp}, the Parliamentarian will inform \gls{ec} of the result of the hearing.

	\end{enumerate}

\section{Sanctions}
	\begin{enumerate}
		\item The board may issue any sanctions.  The following are suggested examples of possible sanctions:
			\begin{enumerate}
				\item Community Service --- The brother may be required to perform a certain number of hours of service to the community.  The exact form of service may or may not be specified.
				\item Cost of repairs --- If the actions of the brother caused damage to property, the brother may be charged for the cost of any repairs and required to assist in repairing the damage.
				\item Counseling --- The brother may be required to seek counseling through the appropriate university office, or an outside agency.
				\item Fines --- The brother may be fined.  Fines will be due to the treasury.
				\item House service --- The brother may be required to perform a cleanup or repair task to benefit the house or grounds.
				\item Letter of apology --- The brother may be required to apologize to the wronged party in writing.
				\item Loss of housing status --- The brother may be made \gls{staggard} for a set period of time or until a certain action has been taken.
				\item Loss of Voting Privileges --- The brother's voting rights may be suspended for a set period of time, or until a certain action has been taken.
				\item Recommendation of Suspension --- The brother may be recommended for suspension to the chapter.
				\item Removal from Office --- If the brother holds an appointed position, the board may remove the brother from office. If the brother holds an elected office, the board may recommend the chapter remove him in accordance with Article V, Section 4 of the \gls{ibl}.
				\item Probation --- The brother may be placed on \gls{prob} for a set period of time or until an action has been taken.
			\end{enumerate}
		\item The Arbiter will ensure the completion of sanctions.
			\begin{enumerate}
				\item If the sanctions are not completed, the Arbiter will bring the brother up for standards.
			\end{enumerate}
	\end{enumerate}

\section{Appeals}
	\begin{enumerate}
		\item A brother may appeal a decision of the board.
        \begin{enumerate}
            \item If the appeal is for reasons outlined in Article XII, Section J, Item 2, Points (a) or (c), the appeal must be submitted within five days of receiving written notification of the results of the hearing.
        \end{enumerate}
		\item A decision can only be appealed for the following reasons:
			\begin{enumerate}
				\item The procedure for the hearing was not followed.
				\item New information is discovered that wasn't available at the time of hearing.
				\item The sanction is inappropriate.
			\end{enumerate}
		\item The appeal will be presented in writing to the Parliamentarian and will state the grounds for the appeal.
		\item \gls{ec} will determine the validity of the appeal.
		\item If the appeal is valid, the confidentiality of the hearing is waived and at the next chapter meeting the appellant will have five minutes to explain the reason for the appeal. The Arbiter will then have five minutes to explain the board's decision. Following the statements, the accused leaves the room and a discussion takes place. A \gls{super-maj} vote is needed to repeal the decision.
		\item If the decision is repealed, the chapter may give new sanctions by \gls{simp-maj} vote.
		\item After the chapter has voted, the accused brother will return to the meeting and will be informed of the decision by the President.
	\end{enumerate}

\section{Academic Review}
	\label{academic-review}
	\begin{enumerate}
		\item An Academic Review shall be held whenever the Scholarship Chair requests a suspension trial in accordance with the duties of their office. 
		\item An Academic Review must be scheduled prior to the associated suspension trial.
		%Since the scholarship bylaws instruct the scholarship chair to request suspension trials for retreat, the academic review needs to be the day before. This don’t seem to be the sort of thing to actually put into the bylaws.
		\item An Academic Review is not a Standards Board Trial.
		% Just to be clear, there is no accuser, there is no accused, there are no charges, there is no plea, and there are no sanctions.
		\item An Academic Review shall be carried out by the Arbiter, Parliamentarian, Scribe, Scholarship Chair, and Standards Board Justices. 
		\item A quorum of Standards Board is required to hold an Academic Review.
		\item Attendance of an Academic Review is mandatory for all members of the board, and the Scholarship Chair.
		\item A justice may be recused from an Academic Review under the following circumstances:
			\begin{enumerate}
				\item Members of the board for whom the Scholarship Chair is requesting a suspension trial will recuse themselves.
				% As it stands this cannot actually happen, as to be brought up by the scholarship chair you have to be below the officer threshold, and as such are automatically removed from your position. This is here just in case it doesnt work like that, and regardless this should still be a thing.
				\item Members of the board will recuse themselves if they feel a conflict of interests exists.
				\item Members of the board may be recused by simple majority vote of the remaining board. The brother under review, the Scholarship Chair, or a member of the board may call for such a vote.
			\end{enumerate}
		\item Process of an Academic Review
			\begin{enumerate}
				\item  Scheduling:
				\begin{enumerate}
					\item The Arbiter will schedule Academic Reviews.
					\item All Academic Reviews for a semester shall be held on the day before that semester’s first chapter meeting.
					\item If members of the board are under academic review, their reviews will be scheduled before reviews of non-members. If the Arbiter is under academic review, their review will be scheduled first.
					% Again not sure if this can happen but alas
					\item The Arbiter will inform the board, Scholarship Chair, and brothers under review of the day’s schedule within three days of the Scholarship Chair requesting academic suspensions.
				\end{enumerate}
				\item  Preparation:
				\begin{enumerate}
					\item The brother under review is not required to attend.
					% It’s not reasonable to try to force people to show up the day before retreat on what might be fairly short notice. (Specifically regarding Spring semester)
					\item Both the brother under review and the Scholarship Chair may provide witnesses and any form of relevant documents, including written statements. If either the brother under review or the Scholarship Chair cannot attend the review, any such documents must be sent to the scribe along with an order of any witnesses to be called.
					\item The Arbiter will call the meeting to order and ask all but unrecused members of the board, the Scholarship Chair, and the brother under review to leave the room.
				\end{enumerate}
				\item Method:
				\begin{enumerate}
					\item The Scholarship Chair and brother under review may give an opening statement, in that order.
					\item The Scholarship Chair may then be asked questions by the brother under review and the board.
					\item The brother under review may then be asked questions by the Scholarship Chair and the board.
					\item The Scholarship Chair and then the brother under review may then present any witnesses or documents. If they are not present, the provided witnesses and documents will be presented in the provided order. After each witnesses gives a statement the Scholarship Chair, brother under review, and board may ask questions of the witness. Documents will remain with the scribe until the conclusion of the review.
					\item The Scholarship Chair and brother under review may give closing statements, in that order.
					\item The Arbiter will excuse the Scholarship Chair and brother under review.
					\item The board will then deliberate. The board may consult any documents and recall any witnesses, the Scholarship Chair, or the brother under review if further questions arise.
					\item The board will vote to recommend the brother under review for suspension, or to not recommend the brother under review for suspension.
					\item The Arbiter will recall the Scholarship Chair and the brother under review and inform them of the boards decision.
					\item If the board voted not to recommend the brother under review for suspension, the Arbiter will instruct the Scholarship Chair to withdraw their request for a suspension trial for the brother under review immediately.
				\end{enumerate}
			\end{enumerate}
		\item If an academic review cannot be held for a brother before the requested suspension trial, the Arbiter will inform the chapter during that trial that no review was held and that no recommendation has been given.
		\item In the event that the board does not recommend the brother under review for suspension the proceedings of the review will be confidential.
		\item In the event that the board recommends the brother under review for suspension the proceedings of the review will not be confidential. Any statements by the Scholarship Chair, brother under review, or witnesses, the content of any presented documents, and the nature of the boards deliberation may be requested and discussed during the associated suspension trial.
	\end{enumerate}

\section{Awards}
	\begin{enumerate}
		\item The board will meet once per month to recognize brothers in the chapter.
		\item The board will give out the following awards once per month:
			\begin{enumerate}
				\item Brother of the Month --- To be given to a brother in recognition of his outstanding service to the chapter in the preceding month.
				\item Officer of the Month --- To be given to an officer in recognition of his outstanding service to the chapter in the preceding month.
			\end{enumerate}
        \item The board will give out the following awards once per semester:
            \begin{enumerate}
                \item Brother of the Semester
                \item Alumnus of the Semester
            \end{enumerate}
        \item The board will nominate brother for Case Western Reserve University Greek Life's awards annually.
		\item A runner-up may be announced for each award.
		\item The board may give other awards at their own discretion. 
		\item The board will vote on all awards to be given.
	\end{enumerate}
	\pagebreak